\section{Resolución}

Tenemos $k$ calles bidireccionales a comparar y buscamos de entre ellas la que minimice el valor del camino más corto entre $s$ y $t$ luego de construirla. Este problema puede ser modelado con un digrafo, donde todo par de puntos está conectado por a lo sumo una arista dirigida. Luego, $s$ y $t$ contarán, o no, con un camino mínimo original el cual buscamos mejorar, o crear, agregando las aristas correspondientes a las calles bidireccionales propuestas. Proponemos entonces el siguiente algoritmo:

\begin{enumerate}
    \item Armamos un digrafo que modele a la ciudad, donde cada punto es un nodo y cada calle unidireccional es una arista. 
    \item Utilizamos el algoritmo de Dijkstra para encontrar los caminos minimos desde $s$ hacia todos los nodos.
    \item Dando vuelta las aristas, utilizamos el algoritmo de Dijkstra para encontrar el camino mínimo de todos los nodos hacia $t$ en el digrafo original.
    \item Por cada arista bidireccional $k_i$ de largo $l(k_i)$ que une a dos puntos $v$ y $w$, y sea $d(u,p)$ la longitud del camino más corto entre los vértices $u$ y $p$, comparamos el mínimo de entre: 
    \begin{itemize}
        \item[--] La longitud del mejor camino hasta ahora entre $s$ y $t$.
        \item[--] Ir desde $s$ hasta $v$, viajar a $w$ y desde $w$ hasta $t$ con costo $d(s,v) + l(k_i) + d(w,t)$.
        \item[--] Ir desde $s$ hasta $w$, viajar a $v$ y desde $v$ hasta $t$ con costo $d(s,w) + l(k_i) + d(v,t)$.
    \end{itemize}
    \item Devolver la longitud mínima de entre todas las halladas.
\end{enumerate}


La idea detrás del algoritmo es considerar todas las calles posibles a construir sin tener que recalcular el nuevo camino mínimo entre $s$ y $t$ con el algoritmo de Dijkstra por cada una. Entonces, luego de obtener los mejores caminos desde $s$ hacia todos los nodos, y desde todos los nodos hacia $t$, podemos rápidamente obtener la longitud del nuevo camino más corto luego de construir la calle $k_i$ con las fórmulas mencionadas en el paso 4. 

\subsection{Implementación}
\vspace{1em}

Presentamos a continuación una posible implmentación de la solución explicada:

\lstinputlisting[mathescape=true, language=pseudo, label=e, caption={Pseudocódigo}]{ej3.pseudo}

\subsection{Demostración de Correctitud}
\vspace{1em}

Queremos ver que el algoritmo devuelve una solución válida y óptima. Por un lado, buscamos probar que la solución que obtiene es uno de los caminos mínimos entre $s$ y $t$ luego de agregar alguna de las $k$ calles bidireccionales posibles a la ciudad. Por otro lado, queremos ver que no existe un resultado mejor. Es decir, para cualquier otro camino mínimo entre $s$ y $t$ con el agregado de una calle a la ciudad, su longitud sera igual o mayor a la devuelta.

\vspace{1em}

Demostremos todo esto haciendo inducción sobre el invariante del algoritmo. Observando el mismo, se puede apreciar que luego de $i$ pasos, se habrán considerado las primeras $i$ calles posibles a construir, y se tiene almacenada la longitud del camino más corto entre $s$ y $t$ luego de haber construido una de ellas. Llamemos la variable que guarda este valor $mn$. 

\vspace{1em}

\textbf{Caso base:} Cuando $i = 0$, no se consideró todavía ninguna calle bidireccional y $mn$ almacena el mejor camino entre $s$ y $t$ en el digrafo original. Entonces, se cumple triviálmente la hipótesis inductiva.

\vspace{1em}

\textbf{Paso Inductivo:} Supongamos que luego de $i$ pasos del algoritmo se consideraron las primeras $i$ calles bidireccionales, y se tiene almacenada la longitud del mejor camino entre $s$ y $t$ habiendo construido una de ellas. Buscamos probar que, luego del paso $i+1$, se consideró la siguiente calle y se actualizó correctamente el valor almacenado.

\vspace{1em}

En este paso, se considera la calle bidireccional $k_{i+1}$ que tiene longitud $l(k_{i+1})$ y conecta a los puntos $v$ y $w$. Luego, el algoritmo toma la menor de entre las siguientes tres longitudes posibles:

\begin{itemize}
        \item La ya almacenada en $mn$.
        \item Ir desde $s$ hasta $v$ con longitud mínima, viajar a $w$, y desde $w$ hasta $t$ con longitud mínima, con costo $d(s,v) + l(k_{i+1}) + d(w,t)$.
        \item Ir desde $s$ hasta $w$ con longitud mínima, viajar a $v$, y desde $v$ hasta $t$ con longitud mínima, con costo $d(s,w) + l(k_{i+1}) + d(v,t)$.
\end{itemize}

De ser $mn$ el menor de los tres, por hipótesis inductiva esta es la longitud del mejor camino mínimo entre $s$ y $t$ hasta el momento, con lo cual se cumple que $mn$ sigue siendo el mejor luego de $i+1$ pasos.

\vspace{1em}

Ahora, pensemos que pasa cuando el mínimo es uno de los dos nuevos caminos que pasan por la calle bidireccional $k_{i+1}$. Supongamos que la longitud del camino que pasa por $k_{i+1}$ desde $v$ hacia $w$ es la mínima de las tres opciones (la demostración es idéntica en el caso en que pase por $k_{i+1}$ desde $w$ hacia $v$). 

\vspace{1em}

Por hipótesis inductiva, como $mn$ es la longitud del mejor camino mínimo hasta ahora, este nuevo camino es mejor que todos los anteriores. Luego, solo queda ver que en el digrafo donde se agregaron las aristas $v \rightarrow w$ y $w \rightarrow v$, este nuevo camino es efectivamente uno mínimo entre $s$ y $t$. Para ello, debemos demostrar que no existe otro camino entre ambos puntos críticos que tenga menor longitud que $d(s,v) + l(k_{i+1}) + d(w,t)$. 

\vspace{1em}

Como este camino es mejor que todos los anteriores hasta este momento, en particular es mejor que el camino mínimo entre $s$ y $t$ del digrafo original. Entonces, como se agregaron solo las dos aristas nuevas $v \rightarrow w$ y $w \rightarrow v$, todo nuevo camino mínimo entre $s$ y $t$ mejor que el original deberá necesariamente pasar por alguna de ellas. Pero de existir uno mejor que los mencionados, significa que iría desde $s$ o $t$ hasta $v$ o $w$ con menor longitud que la mínima posible, lo cual es absurdo porque estas distancias fueron calculadas con el algoritmo de Dijkstra, y por ende no existen caminos más cortos. Luego, el camino $s \rightarrow v \rightarrow w \rightarrow t$ es mejor que todos los anteriores y es el mínimo en el digrafo del paso actual, con lo cual se cumple que es el mejor luego de $i+1$ pasos.

\vspace{1em}

Queda entonces demostrado que en el paso $k_{i+1}$ se consideran las primeras $i+1$ calles bidireccionales y se almancena en $mn$ la longitud del mejor camino mínimo entre $s$ y $t$ habiendo construido una de ellas. Al finalizar el algoritmo, se habrán considerado todas las calles propuestas y se devolverá efectivamente la longitud del camino mínimo mas corto posible. También observar que, de no existir camino entre $s$ y $t$ incluso agregando algunas de las $k$ aristas, el mínimo calculado en cada paso será siempre infinito, y por ende el algoritmo reconocerá correctamente que debe de devolver -1.

\subsection{Análisis de la Complejidad}
\vspace{1em}

La complejidad del algoritmo presentado es la siguiente: 

\begin{itemize}
    \item Obtener input \qquad \qquad \qquad \qquad \qquad \quad \quad \ \ $\rightarrow O(M + K)$
    \item Aplicar dijkstra \qquad \qquad \qquad \qquad \qquad \qquad $\rightarrow O(dijkstra)$
    \item Buscar camino mínimo con nuevos caminos $\rightarrow O(K)$
\end{itemize}

Como se puede apreciar, la complejidad total del algoritmo es O(m + k) + O(dijkstra) por lo que depende estrechamente del algoritmo de dijkstra que se utilice. A continuación presentaremos dos implementaciones diferentes.

\vspace{1em}

La idea del algoritmo de dijkstra, en resumen, es a partir de un nodo inicial, llevar cuenta de a que nodos nuevos podemos acceder y tomar 'golosamente' el camino que nos permita acceder a un nodo desconocido de manera más barata. Luego a través de este nuevo nodo podremos acceder a otros, por lo que hay que actualizar nuestra estructura, incluyendo todas las nuevas opciones de nodos a las que podemos acceder sumándole la distancia que nos costó llegar al nodo padre. Y así repetir este ciclo hasta que no podamos acceder a ningún nodo nuevo. 

\pagebreak
\lstinputlisting[mathescape=true, language=pseudo, label=e, caption={Pseudocódigo para dijkstra implementado con heap}]{dijkstra_heap.pseudo}


En esta primera implementación la estructura con la que representamos los nodos accedibles en cada iteración del ciclo es un heap. 
Se puede observar que se hacen $d(v)$ inserciones al heap, para cada $v \in V$, que sabemos que en total suman $2 * M$, y el while corre hasta que el heap se encuentre vacío, por lo que además haremos $2 * M$ extracciones del heap. Ambas operaciones, inserción y extracción tienen una complejidad de $O(log(M))$. Podemos concluir entonces, que la complejidad del algoritmo es $O(M * log(M))$.
Cabe aclarar que si el grafo es denso es decir $M \approx n^2$ entonces el algoritmo realizará $O(N^2 * log(N^2))$. 


\lstinputlisting[mathescape=true, language=pseudo, label=e, caption={Pseudocódigo para dijkstra implementado para grafos densos}]{dijkstra_dense.pseudo}

En esta implementación la estructura con la que representamos los nodos accedibles en cada iteración del ciclo es un vector, donde se almacena el mínimo costo para acceder a cada nodo. 
Analizando la complejidad, notamos que hay exactamente n ciclos y en cada uno de ellos, por un lado recorremos el vector de n posiciones y además realizamos $d(v)$ operaciones para cada $v \in V$, que como mencionamos previamente, la suma total es $2 * M$. Es decir que la complejidad del algoritmo es $O(N^2 + M)$ pero además sabemos que $M \leq N^2$, o sea que la complejidad del algoritmo es $O(N^2)$.
Como observamos antes, esta implementación es más conveniente que la otra si el grafo es suficientemenete denso. 


