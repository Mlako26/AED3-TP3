\section{Introducción}

El \textit{problema de los modems} consiste en calcular la mejor forma de brindarles acceso a internet a un conjunto de oficinas. Dadas $N$ oficinas, se tienen $W$ modems para instalar en ellas, con $W < N$. El problema surge de que la cantidad de modems es menor, con lo cual se deben de conectar con cables algunas oficinas entre sí para compartirse conexión. Hay dos tipos de cables: los UTP y los de fibra óptica, con precios por centímetro $U$ y $V$ respectivamente. La diferencia entre estos cables es que los UTP suelen ser más baratos, $U \leq V$, aunque solo se pueden utilizar para conectar dos oficinas que estén a distancia $R$ o menor.

\vspace{1em}

Entonces, dadas las posiciones de las $N$ oficinas en un eje cartesiano medido en centímetros, la cantidad $W$ de modems, los precios $U$ y $V$ de ambos cables y la distancia $R$, debemos calcular el mínimo costo en cables necesario para conectar a todas las oficinas. Además, se tiene que aclarar cuánto se gasta en cables UTP y de fibra óptica, respectivamente\footnote{Restricciones a los parámetros: \\ \indent \indent $1 \leq W < N \leq 1000 \ \wedge \ 1 \leq U \leq V \leq 10 \ \wedge \ 1 \leq R \leq 10000 \ \wedge \  -10000 \leq x_i, y_i \leq 10000. $}.

\vspace{1em}

