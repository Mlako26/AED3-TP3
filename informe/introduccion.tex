\section{Introducción}

\textit{Mejorando el tráfico} es un problema en el cual queremos mejorar la distancia entre dos puntos \texttt{s} y \texttt{t} de una ciudad. 

\vspace{1em}

Entonces, dadas las posiciones de las $N$ oficinas en un eje cartesiano medido en centímetros, la cantidad $W$ de modems, los precios $U$ y $V$ de ambos cables y la distancia $R$, debemos calcular el mínimo costo en cables necesario para conectar a todas las oficinas. Además, se tiene que aclarar cuánto se gasta en cables UTP y de fibra óptica, respectivamente\footnote{Restricciones a los parámetros: \\ \indent \indent $1 \leq W < N \leq 1000 \ \wedge \ 1 \leq U \leq V \leq 10 \ \wedge \ 1 \leq R \leq 10000 \ \wedge \  -10000 \leq x_i, y_i \leq 10000. $}.

\vspace{1em}

