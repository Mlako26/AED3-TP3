\section{Introducción}

\textit{Mejorando el tráfico} es un problema en el cual buscamos acortar la distancia entre dos puntos de una ciudad. El mapa de ésta consta de $n$ puntos diferentes unidos de a pares por $m$ calles unidireccionales. Luego, dados dos puntos críticos $s$ y $t$, se propone una lista de $k$ nuevas calles bidireccionales a construir con el propósito de reducir la longitud del camino más corto entre ambos. El objetivo de este problema es encontrar de entre ellas la que minimice la distancia resultante del camino mínimo entre $s$ y $t$.

\vspace{1em}

Entonces, dadas las $m$ calles unidireccionales que unen a los diferentes $n$ puntos de la ciudad, los puntos críticos $s$ y $t$, y las $k$ calles bidireccionales propuestas, buscamos encontrar la mejor de ellas y devolver la longitud del camino mínimo resultante entre $s$ y $t$ luego de construirla. En caso de no existir un camino entre ambos puntos críticos, se debe de devolver el valor $-\ 1$ que lo indica\footnote{Restricciones a los parámetros: \\ \indent \indent $n \leq 10000, m \leq 100000, k \leq 300, 1 \leq s, t \leq n, 0 < l_i, q_i \leq 1000$, con $l_i$ y $q_i$ la longitud de las calles de la ciudad \indent \indent y de las calles propuestas respectivamente.}.

\vspace{1em}